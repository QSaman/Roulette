\documentclass{book} 

\usepackage{graphicx}
\usepackage{amsmath}
\usepackage{algorithm}
\usepackage{algpseudocode}
\usepackage{float}
\usepackage{hyperref}
\usepackage{placeins}
\usepackage{amssymb}
\usepackage{listings}
\usepackage{hyperref}

\usepackage{tikz}
\usetikzlibrary{arrows}
\usetikzlibrary{snakes}
\usetikzlibrary{decorations.pathmorphing}

\hypersetup{colorlinks=true,linkcolor=blue, linktocpage}

\title{Roulette}
\author{Saman Saadi}
\date{}

\lstdefinestyle{customcpp}{
	belowcaptionskip=1\baselineskip,
	breaklines=true,
	frame=L,
	xleftmargin=\parindent,
	language=C++,
	frame=single,
	showstringspaces=false,
	basicstyle=\footnotesize\ttfamily,
	keywordstyle=\bfseries\color{green!40!black},
	commentstyle=\itshape\color{purple!40!black},
	identifierstyle=\color{black},
	stringstyle=\color{orange},
	emph={int,char,double,float,unsigned, auto},
	emphstyle={\color{blue}}
}

\lstset{escapechar=@,style=customcpp}

\begin{document}
	\frontmatter
	\maketitle
	\mainmatter
	\section{Summary}
	Let's assume we want to bet for the round $i^{th}$ in a game and our target profit is \$$p$ for each round until we win. So far we lost in the previous rounds. We call The amount of bet in $i^{th}$ round, $b_i$:
	\begin{equation*}
		b_i = 2 \times b_{i - 1}
	\end{equation*}
	Or we can say:
	\begin{equation*}
		b_i = (2^{i - 1}) \times p
	\end{equation*}
	In other words, if our previous bet was $x$, we should bet $2 \times x$ this time!
	\par If we want to afford to lose ${n - 1}$ times in a row and win in the $n^{th}$ round, our budget should be:
	\begin{equation*}
		budget = (2^n - 1) \times p
	\end{equation*}
	\section{Expected value}
	The American roulette has 37 pockets and the Canadian has 36. We use $n$ for the number of pockets in the wheel. Let's assume we bet on $c$ numbers. For example for betting on number $1$, $c = 1$ and for a red number $c = 18$. The payout is $36 - c$ to $c$. That means for every \$$c$ bet, we get \$$36 - \$c$. We also receive the original \$$c$. So our total balance for \$$1$ if we would win is:
	\begin{equation*}
		\begin{split}
			&\frac{36 - c + c}{c} \\
			&= \frac{36}{c}
		\end{split}
	\end{equation*}
	If we lose, we lose \$$1$. So the expected value is:
	\begin{equation*}
		\begin{split}
			E &= \frac{36 - c}{c} \times \frac{c}{n} - \frac{c}{c} \times \frac{n - c}{n} \\
			&= \frac{36 - c}{n} - \frac{n - c}{n} \\
			&= \frac{36 - n}{n} \\
			&= \frac{36}{n} - 1
		\end{split}
	\end{equation*}
	As you can see the expected value is not related to our choice! If we have 36 pockets, it's zero and we can call it a fair game. In Canada it's $\frac{36}{37} - 1 = -0.027$ and in the US it's $\frac{36}{38} - 1 = -0.053$. In other words, in Canada on average we should lose 3 cents per dollar and in US, 5 cents per dollar.
	\par As an example, let's assume we have 3 chips with the value of \$$\frac{1}{3}$. We bet one of them on black, the other on an even number and the other on a number greater than 18. Note that all of our bets are in the same round. we assume $p = \frac{c}{n}$. The expected value is:
	\begin{equation*}
		\begin{split}
			E &= \underbrace{\frac{3}{3} \times p^3}_{\text{3 wins}} \\
			  &+ \underbrace{(\frac{2}{3} - \frac{1}{3}) \times \binom{3}{2} \times p^2 \times (1 - p)}_{\text{2 wins and 1 loss}} \\
			  &+ \underbrace{(\frac{1}{3} - \frac{2}{3}) \times \binom{3}{2} \times p \times (1 - p)^2}_{\text{1 win and 2 losses}} \\
			  &+ \underbrace{(-\frac{3}{3}) \times (1 - p)^3}_{\text{3 losses}} \\
			  &= p^3 + \frac{1}{3} \times 3 \times p^2 \times (1 - p) - \frac{1}{3} \times 3 \times p \times (1 - p)^2 - (1 - p)^3 \\
			  &= p^3 + p^2 \times (1 - p) - p \times (1 - p)^2 - (1 - p)^3 \\
			  &= 2 \times p - 1
		\end{split}
	\end{equation*}
	Note that we should multiply the probability by $\binom{3}{2}$ when we have 2 wins and 1 loss. Because the status of the chips is one of the three sequences ($W$ means win and $L$ means loss):
	\begin{equation*}
		\begin{split}
		WWL \\
		WLW \\
		WWL \\
		\end{split}		
	\end{equation*}
	The probablity of each of these sequenes is $p^2 \times (1 - p)$. We use the same logic for 2 losses and 1 win.
	\par Now let's assume we have 2 chips with the value of \$$\frac{1}{2}$. We bet one of them on black and the other on an even number in the same round. The expected value is:
	\begin{equation*}
		\begin{split}
			E &= (\frac{1}{2} + \frac{1}{2}) \times p^2 \\
			&+ (\frac{1}{2} - \frac{1}{2}) \times \binom{2}{1} \times p * (1 - p) \\
			&+(-\frac{1}{2} - \frac{1}{2}) \times (1 - p)^2 \\
			&= p^2 - (1 - p)^2 \\
			&= 2 \times p - 1
		\end{split}
	\end{equation*}
	\par Now let's assume we have a chip of value \$$1$ and we bet on a black number. The expected value is:
	\begin{equation*}
		\begin{split}
			E &= 1 \times p - 1 \times (1 - p) \\
			&= 2 \times p - 1			
		\end{split}
	\end{equation*}
	So it doesn't matter what we choose, we are going to lose $2 \times p - 1 = -0.027$ per dollar in Canada! It's almost $3$ cents per dollar.
	\section{Chance of winning}
	In each round the change of winning in Canada is $\frac{18}{37} \simeq 49\%$ and the chance of loss is $\frac{19}{37} \simeq 51\%$. In the US the chance of winning is $\frac{18}{38} \simeq 47\%$ and the chance of loss is $\frac{20}{38} \simeq 53\%$.
	\par Please refer to this \href{https://www.youtube.com/watch?v=56iFMY8QW2k}{MIT lecture} For a more detailed explanation.
	\section{General case}
	Let's assume we are in round $i^{th}$ of a game. We can calculate the balance for round $i$ that we call it $b_i$. We want to make $p$ unrealized profit/loss for this game, assuming we eventually win in this round
	\subsection{Mathematical induciton}
	We define $b_i$ as the amount of bet for round $i$ in such a way that if we win round $i$, our total profit for this game would be $p$. We use mathematical induciton and we assume we know how to solve $b_{i - 1}$.
	\subsubsection{When $b_{i - 1} < 0$}
	Let's assume we will win at the end of round $i$ and our profit will be $p$. In round $i - 1$ we lost $b_{i - 1}$ so we just need to spend $b_{i - 1}$ to cancel it. We use mathematical hypotheis and spend another $b_{i - 1}$ to cancel losses for rounds $1$ to $i - 2$ and gain $p$ at the end of the round (assuming we will win): 
	\begin{equation*}
		\begin{split}
			b_i &= \sum_{j = 0}^{i - 1}{b_j} - p \\
			&= \underbrace{\sum_{j = 0}^{i - 2}{b_j} - p}_{b_{i - 1}} + b_{i - 1} \\
			&= 2 \times b_{i - 1}
		\end{split}
	\end{equation*}
	\subsubsection{When $b_{i - 1} \ge 0$}
	Note that if $b_{i - 1} \ge p$, we've already achieved our goal and no need to continue the game for the profit $p$. We may increase the target profit to a more ambition one later.
	\par Let's assume $b_{i - 1} > 0 \land b_{i - 1} < p$ and we will win at the end of round $i$ and our profit will be $p$:
	\begin{equation*}
		\begin{split}
			b_{i - 1} + \underbrace{x}_{bet} \underbrace{- 2 \times x}_{rewards} &=  p \\
			\implies x &= b_{i - 1} - p
		\end{split}
	\end{equation*}
	So we just need to bet on $b_{i - 1} - p$ and if we win this round our profit will be $p$.
	\subsection{General Formula}
	\begin{equation*}
		b_i = \begin{cases}
			2 \times b_{i - 1} & b_{i - 1} < 0 \\
			b_{i - 1} - p & b_{i - 1} \ge 0 \land b_{i - 1} < p \\
			b_{i - 1} & b_{i - 1} \ge 0 \land b_{i - 1} \ge p \\
			b_0 -p & i = 1
		\end{cases}	
	\end{equation*}
	Note that we define $b_1 = b_0 - p$. Usually $b_0 = 0$, but if in the round $k$ we decide to change $p$ to $p^{\prime}$. Then we can initiate a new game with $b_0 = b_k$ and profit $p^\prime$.  
	\section{Losing all previous rounds}
	Let's consider the worst case scenario and we are going to lose in $n$ rounds in a row:
	\begin{equation*}
		\begin{split}
			b_0 &= 0 \\
			b_1 &= b_0 - p = -p \\
			b_2 &= 2 \times b_1 = -2p \\
			b_3 &= 2 \times b_2 = -4p \\
			b_4 &= 2 \times b_3 = -8p \\
			&\vdots \\
			b_{n} &= 2 \times b_{n - 1} = -2^{n - 1} \times p
	     \end{split}
	\end{equation*}
	\section{Budget for $n$ rounds}
	Now let's calculate what should be our budget if we can afford to lose in ${n -1}$ rounds and win at round $n$. We assume $r_i$ is our loss in round $i^{th}$ for $1 \le i \le n - 1$
	\begin{equation*}
		\begin{split}
			budget &= -\sum_{i = 0}^{n}{b_i} \\
			&= \sum_{i = 0}^{n}{2^{i - 1} \times p} \\
			&= (2^n - 1) \times p
		\end{split}
	\end{equation*}
	Note that we use the following formula to get the budget:
	\begin{equation*}
		\sum_{i = 0}^{n}{2^i} = 2^{n + 1} - 1
	\end{equation*}
\end{document}
